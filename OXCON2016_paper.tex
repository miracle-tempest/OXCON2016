%
% Paper to be submitted to the 2016 Oxford Comic Con
% (OXCON 2016), Oxford, England, 5--6 March 2016.
%
% For information on this file please contact
% Miranda K. Loughry
% at Tel. +1 303 221 4380 (time zone GMT minus
% 7 hours) or Email: miranda.loughry@gmail.com
%


\documentclass[10pt,a4paper]{article}

\usepackage[english,british]{babel}
\usepackage{graphicx}
\usepackage{url}
\usepackage[affil-it]{authblk}

\begin{document}

\title{The Modern \emph{Gilgamesh}: \\
Jungian Archetypes in \emph{Doctor Who}}

\author{Miranda Loughry%
\thanks{Corresponding author's address: \texttt{miranda.loughry@gmail.com}}}
\affil{St Mary's Academy \\
Englewood, Colorado 80113 \\
USA}

\date{\today}

\maketitle

\begin{abstract}
Since the Agricultural Revolution, storytelling has evolved from a means to caution future generations into a ideal medium to reveal humanity's most primal fears and desires; stories discussing the essence of humanity, the inevitability of mortality, humankind's
relationship with nature, and the role of the
divine appear throughout history. From the
\emph{Epic of Gilgamesh} to \emph{Doctor Who}
and \emph{Forbidden Planet}, many of humanity's
most enduring stories are built on the same
fundamental patterns. Analyzing these
cross-cultural, cross-generational motifs
through the lens of modern science fiction
reveals powerful commentary on the nature of
humanity. Like Gilgamesh, \emph{Doctor Who}
never was about the Doctor.
\end{abstract}

\section{Introduction}
%
%Archetypes are pancultural, pangenerational manifestations of the collective unconcious.
%

\section{Background}
\subsection{The \emph{Epic of Gilgamesh}}
%
%Introduction to Mesopotamia and the \emph{Epic of Gilgamesh}
%The Fertile Crscent is the craddle of civilization, oldest civilization in the world. The \emph{Epic of GIlgamesh} is the oldest known work of Western fiction. It stands out because it is a narrative, a long poem, when all other works at the time were inventories/financial transactions. It was written in cuneiform. Some of the tablets are missing. Storytelling originated in Mesopotamia alongside the written word. Discusses themes relevent even today.
%
Agriculture was born about ten thousand years ago in a region of southwest Asia near the Mediterranean Sea. Here, in what is now called the Fertile Crescent, the floodwaters of the Nile, the Tigris, and the Euphrates rivers annually deposited rich nutrients in the soil, creating a sickle-shaped ribbon of arable land in the otherwise arid landscape. As techniques for cultivating the land became more efficient, grain surpluses necessitated less emphasis on food production, and specialized occupations not directly related to agriculture or animal husbandry arose. With this new system for division of labor, scribes, priests, craftsmen, artisans, and mercenaries came into being, and civilization followed in their wake \cite{nelson2015}. 

Nestled between the Tigris and Euphrates rivers was one of the greatest of these early river valley civilizations: Mesopotamia. Each of the largely sovereign city-states contained within Mesopotamia had certain cultural idiosyncrasies, and these religious, ideological, and linguistic differences compounded to create a mosaic of rich art and literature that characterizes the region \cite{van1997ancient}. 

Mesopotamia’s stories were conveyed orally, often for recreation in trading caravans and in noble courts as well as in the cities, but most were also written on clay tablets in cuneiform \cite{dalley2000myths}. Though only fragments of many of these tablets have been recovered, the oldest and most complete collection known today belongs to the royal library of Ashurbanipal, the last king of the Neo-Assyrian Empire, who amassed a collection of literary, scientific, religious, and historical documents. While the library was founded in the seventh century, many of the works it chronicles are thousands of years older, translated from ancient Sumerian into Akkadian by Ashurbanipal’s servants on their travels through Babylon, Nippur, and Uruk. Among these works is the \emph{Epic of Gilgamesh}: the oldest known work of Western literature \cite{sandars1972epic}.
%
%Gilgamesh is the fifth king of Uruk after the great flood. He built the city's walls. He's two thirds god and one third human. He is the greatest warrior ever to live, but he oppressed his people, etc.. It is the story of Gilgamesh andEnkidu becoming human together.
%

\subsection{\emph{Doctor Who}}
%
%Introduction to modern science fiction and \emph{Doctor Who}
%Science fiction is by nature an experimental genre. It allows its creators to challenge social mores, discussing many ancient themes. 
%
%\emph{Doctor Who} is a long-running BBC television program about a functionally immortal Time Lord who travels through time and space fighting monsters and learning to be human with the help of his companions. The Doctor was forced to ommit genocide in order to end the Time War.
%

\section{Comparative Archetypes}
\subsection{The Hero's Journey}
%
%Out of normalcy and into the realm of the supernatural, wherein they must fight monsters/overcome impossible odds before returning with a bounty, home.
%

\subsection{The Wild Man}
%
%Ran with the animals, drank from their streams, introduced to society, rejected by the animals. 
%

\subsection{Femininity and Misogyny}
\subsubsection{The Wise Woman}
%
%TARDIS=Ninsum
%

\subsubsection{The Seductress}
%
%Missy=Prostitute (Not sectualized, but distracted the Doctor from his goal)
%

\subsubsection{The Beloved}
%
%River≠Anyone in the \emph{Epic of Gilgamesh}, because it's mysogynistic
%

\subsection{Immortality and the Afterlfie}
%
%Regenration isn't permenant, just as the snake eats the flower. The afterlife is a physical place, but you cannot thwart it.
%

\section{Analysis}
%
%The Doctor and Gilgamesh are both given ultimate power, but are ultimately not qualified to weild it.
%
%Fatal flaw=hubris
%

\section{Summary and Conclusion}
%
%Relevence restated
%

\section{Acknowledgements}
%
%Greateful acknowledgement is made to Dr. Joe Loughry, Kathryn Schmidt, Megan Gorman, Pamela Reich, Peter Sniegowski, Don Nelson, Lisa Weinberg, Anne Evans, and Kathryn MacNammee for insightful comments and mentorship throughout this process.
%

\bibliographystyle{plain}


\bibliography{references.bib}

\medskip\noindent{\scriptsize Build \input{counters/paper_build_counter.txt}



\end{document}