%
% Paper to be submitted to the 2016 Oxford Comic Con
% (OXCON 2016), Oxford, England, 5--6 March 2016.
%
% For information on this file please contact
% Miranda K. Loughry
% at Tel. +1 303 221 4380 (time zone GMT minus
% 7 hours) or Email: miranda.loughry@gmail.com
%


\documentclass[10pt,a4paper]{article}

\usepackage[english,british]{babel}
\usepackage{graphicx}
\usepackage{url}
\usepackage[affil-it]{authblk}

\begin{document}

\title{The Modern \emph{Gilgamesh}: \\
Jungian Archetypes in \emph{Doctor Who}}

\author{Miranda Loughry%
\thanks{Corresponding author's address: \texttt{miranda.loughry@gmail.com}}}
\affil{St Mary's Academy \\
Englewood, Colorado 80113 \\
USA}

\date{\today}

\maketitle

\begin{abstract}
Since the Agricultural Revolution, storytelling
has evolved from a means to caution future
generations into a ideal medium to reveal
humanity's most primal fears and desires;
stories discussing the essence of humanity,
the inevitability of mortality, humankind's
relationship with nature, and the role of the
divine appear throughout history. From the
\emph{Epic of Gilgamesh} to \emph{Doctor Who}
and \emph{Forbidden Planet}, many of humanity's
most enduring stories are built on the same
fundamental patterns. Analyzing these
cross-cultural, cross-generational motifs
through the lens of modern science fiction
reveals powerful commentary on the nature of
humanity. Like Gilgamesh, \emph{Doctor Who}
never was about the Doctor.
\end{abstract}

\section{Introduction}
%Ode to Storytelling
Science fiction is by nature an experimental genre; it allows creators to merge the realm of the factual and that of the fantastic, weaving stories of worlds that are sometimes beneficially technologically augmented, but more often are cursed by the abuse of that technology. It is an ideal medium through which authors can challenge societys often warped morality, as it thrives on the development of new ideas. Science fiction adresses questions fundamental to the human condition, such as the nature of mortality, the role of the divine, civil vs. moral law, civilization vs. savagery, society’s dichotomous depictions of femininity, and ultimately what it means to be human. These are questions that reoccur throughout human history, inherited across generations first through the oral tradition, then through written legends, and in modernity through film alongside literature. It is science fiction’s capacity to seamlessly meld contemporary relevance and demonstration of essential human psychology that makes it so alluring to readers globally and intergenerationally.

%Ode to archetypes



\subsection{Working Thesis}
The shared archetypes in the \emph{Epic of Gilgamesh} and \emph{Doctor Who} reveal humanity’s boundless interconnectivity, fostering  creative collaboration.
   
\section{Archetypes in the \emph{Epic of Gilgamesh}}
\subsection{The \emph{Epic of Gilgamesh}}
The \emph{Epic of Gilgamesh} recounts the story of Gilgamesh, the fifth king of Uruk after the great flood (first (?) Western example of archetype of the deluge, later continued with Noah). Gilgamesh, a demigod, rules draconianly, taxing (?) his people and claiming all the virgin brides as his own. He dreams that a star falls from heaven, and his mother/counsellor/dream interpreter Ninsum, and she tells him the star represents his equal. Lo and behold, a hunter tells Gilgamesh that there's a wildman foiling all his traps (humanity feeling threatened by nature, destroying their income), so Gilgamesh sends a prostitute to induct Enkidu (the wildman) into civilization. Enkidu sleeps with her (the seductress), he's made aware of oldness, and the animals reject him. He goes to Uruk with the prostitute, and Gilgamesh and he become great friends. At one point, Enkidu attacks Gilgamesh for sleeping with a new bride, but they laugh/hug/kiss each other as brothers and go on their merry way. Eventually, the two slay Humbaba, a fire breathing dragon/guardien of the cedar forest (described in terms of nature). Ishtar, Uruk's patron goddess (bimbo/Madonna) wants to marry Gilgamesh, but he refuses, so she brings down the Bull of Heaven on the city (drought/plague) in an effort to kill everyone. Gilgamesh and Enkidu slay that too, but the gods are displeased and make Enkidu ill in revenge. He dies (the direct result of Gilgamesh's fatal flaw: hubris), and Gilgamesh goes mad, dresses himself in dogs' hyde, and ventures into the land of the dead in an effort to find to secret of immortal life. He ventures through the land where the sun rises, meets the scorpian people (guardians of the afterlife), and eventually makes a voyage across a river into the land of the dead to meet with the man who survived the deluge (Noah archetype). He gives Gilgamesh instructions to find a flow which grants the elixir of eternal life, Gilgamesh finds it under a lake, but a snake eats it while it's sitting on the shore, thereby returning the gods' property to them. The last tablet is missing, but Gilgamesh goes back to Uruk empty-handed, rules for a number of years, and dies as a mortal.

\section{Archetypes in \emph{Doctor Who}}
The Doctor is a Time Lord (a member of a humanoid species with two hearts that regenerates instead of dying). Years ago, he was forced to commit genocide in order to end the Time War, which was perpetualy being fought at great expense of lives (Daleks vs. Time Lords). He chose to exterminate all his own people alongside the enemy in order to save the rest of the universe. Ever since, he's wandered about fighting monsters and acquiring companions as he goes. He introduces his companions (normally young human women) to the greater community of the universe, and after that, they die/become estranged from conventional human life.

\section{Juxtaposition and Analysis}     
\subsection{The Hero's Journey}
Both Gilgamesh and the Doctor must venture from the realm of normalcy through an extraordinary land, fighting monsters, etc.. Greek: anagnorisis and peripety.

\subsection{The Wildman}
    Gilgamesh's companion Enkidu and the Doctor's companions both fulfil the archetype of the wildman, in that they are both naive (Enkidu of oldness and the companions of the greater community of the universe). They represent humanity's primal past (commentary on humanity's still savage nature, especially with the companions), and their induction into society (Enkidu by Gilgamesh through the prostitute and the companions through the Doctor's travels) directly results in their deaths (Enkidu damned by the gods for killing Humbaba, the companions in various ways: Amy and Rory by Weeping Angels, zapped into the past and made to live to death; Danny by Cybermen; Clara by the raven; many one-episode companions like Rita, Astrid, etc.) or else their demises (Rose trapped eternally in a parallel universe; Donna forced to forget her time with the Doctor; Me forced to an eternal).
    
\subsection{The Eternal Battle}
The Doctor fights the Time War; Gilgamesh fights the war against mortality.
\subsection{Depictions of Femininity}
\subsubsection{The Seductress}
    Ishtar, the barmaid, and Missy 
    
    Gilgamesh meets various seductress figures who try to lure himaway from his journey, offering to feed/clothe/please him. Mssy isn't sexualized as is the barmaid, but she does lure him away from his journey, try to pull him off the path.

\subsubsection{The Wise Woman}
    Ninsum and the TARDIS
    
    Gilgamesh doesn't listen to Ninsum, and his friend dies because of his folly. The TARDIS always takes the Doctor where he needs to go, and when he defies her, things turn out badly (i.e. Someone normally dies).
    
    In the \emph{Epic of Gilgamesh}, Gilgamesh hires a prostitute to make Enkdu aware of oldness, and he curses her on his death bed. The companions are alagorical for Enkidu, but they don't cure the TARDIS, who introduces them to the universe. 
    
\subsection{The Afterlife}
    The flower of eternal life and regeneration. Both aren't absolute. Both must be curated/moderated, either by the snake or by the Sisterhood.

\subsection{Anagnorisis, Peripety, and Fall from Grace}
    Both Gilgamesh and the Doctor have a fatal flaw: hubris. In this way, they're both tragic heros.
    
    
\section{Summary and Conclusion}

Words\ldots

\section{Acknowledgements}

Dr. Robert Joseph Loughry\ldots

\bibliographystyle{plain}

\bibliography{references.bib}

\medskip\noindent{\scriptsize Build \input{counters/paper_build_counter.txt}



\end{document}