%
% Paper to be submitted to the 2016 Oxford Comic Con
% (OXCON 2016), Oxford, England, 5--6 March 2016.
%
% For information on this file please contact
% Miranda K. Loughry
% at Tel. +1 303 221 4380 (time zone GMT minus
% 7 hours) or Email: miranda.loughry@gmail.com
%


\documentclass[10pt,a4paper]{article}

\usepackage[english,british]{babel}
\usepackage{graphicx}
\usepackage{url}
\usepackage[affil-it]{authblk}

\begin{document}

\title{The Modern \emph{Gilgamesh}: \\
Jungian Archetypes in \emph{Doctor Who}}

\author{Miranda Loughry%
\thanks{Corresponding author's address: \texttt{miranda.loughry@gmail.com}}}
\affil{St Mary's Academy \\
Englewood, Colorado 80113 \\
USA}

\date{\today}

\maketitle

\begin{abstract}
Since the Agricultural Revolution, storytelling has evolved from a means to caution future generations into a ideal medium to reveal humanity's most primal fears and desires; stories discussing the essence of humanity, the inevitability of mortality, humankind's relationship with nature, and the role of the divine appear throughout history. From the \emph{Epic of Gilgamesh} to \emph{Doctor Who} and \emph{Forbidden Planet}, many of humanity's most enduring stories are built on the same fundamental patterns. Analyzing these cross-cultural, cross-generational motifs through the lens of modern science fiction reveals powerful commentary on the nature of humanity. Like Gilgamesh, \emph{Doctor Who} never was about the Doctor.
\end{abstract}

\section{Introduction}
%
%Archetypes are pancultural, pangenerational manifestations of the collective unconcious.
%

Certain mythological motifs permeate human culture. Legends of the primordial chaos, the fire-bringer, the trickster, the deluge, the mother goddess, the hero’s journey, and many others exist in every known cultural tradition [CITATION NEEDED]. Though the animistic respect for nature that is the patron ethos of many tribal cultures is often drowned out by the effects of globalization [CITATION NEEDED], the pantheon of our dreams endures, and thus the fundamental elements of fairytale remain constant throughout folk and popular culture.

The existence of such patterns refutes renowned psychoanalyst Sigmund Freud’s commonly accepted assertion that the unconscious is strictly the gathering-place of the mind’s repressed contents [CITATION NEEDED]. Rather, it aligns with the work of Carl Jung, a protégé of Freud who professed that the unconscious contains universal elements that connect all individuals \cite{jung2014archetypes}. While psychological complexes are the domain of the relatively superficial "personal unconscious,” the collective unconscious is responsible for similarities in intercultural behavioral patterns, material culture, ideology, and mythology [CITATION NEEDED]. Archetypes are these pancultural, pangenerational manifestations of the collective unconscious in the form of patterns in art and literature  \cite{jung2014archetypes}.

\section{Background}
\subsection{The History of Mesopotamia}
%
%Introduction to Mesopotamia and the \emph{Epic of Gilgamesh}
%The Fertile Crscent is the craddle of civilization, oldest civilization in the world. The \emph{Epic of GIlgamesh} is the oldest known work of Western fiction. It stands out because it is a narrative, a long poem, when all other works at the time were inventories/financial transactions. It was written in cuneiform. Some of the tablets are missing. Storytelling originated in Mesopotamia alongside the written word. Discusses themes relevent even today.
%
Agriculture was born about ten thousand years ago in a region of southwest Asia near the Mediterranean Sea.\footnote{Agriculture also evolved relatively simultaneously in various hearth regions around the world: Mesoamerica, where wild grasses were domesticated into maize; the Yangtze region of southeast Asia, where rice was cultivated; and the lowlands and highlands of South America, where peanuts, potatoes, and manioc were grown \cite{brown2009complex}.} Here, in what is now called the Fertile Crescent, the floodwaters of the Nile, the Tigris, and the Euphrates rivers annually deposited rich nutrients in the soil, creating a sickle-shaped ribbon of arable land in the otherwise arid landscape\cite{nelson2015}. As techniques for cultivating the land became more efficient, grain surpluses necessitated less emphasis on food production, and specialized occupations not directly related to agriculture or animal husbandry arose.\cite{brown2009complex} With this new system for division of labor, scribes, priests, craftsmen, artisans, and mercenaries came into being, and civilization followed in their wake \cite{nelson2015}. 

Nestled between the Tigris and Euphrates rivers was one of the greatest of these early river valley civilizations: Mesopotamia. Each of the largely sovereign city-states contained within Mesopotamia had certain cultural idiosyncrasies, and these religious, ideological, and linguistic differences compounded to create a mosaic of rich art and literature that characterizes the region \cite{van1997ancient}. 

Mesopotamia’s stories were conveyed orally, often for recreation in trading caravans and in noble courts as well as in the cities, but most were also written on clay tablets in cuneiform \cite{dalley2000myths}. Though only fragments of many of these tablets have been recovered, the oldest and most complete collection known today belongs to the royal library of Ashurbanipal, the last king of the Neo-Assyrian Empire, who amassed a collection of literary, scientific, religious, and historical documents. While the library was founded in the seventh century, many of the works it chronicles are thousands of years older, translated from ancient Sumerian into Akkadian by Ashurbanipal’s servants on their travels through Babylon, Nippur, and Uruk. Among these works is the \emph{Epic of Gilgamesh}: the oldest known work of Western literature \cite{sandars1972epic}.

Written in the third millennium BCE, \emph{Gilgamesh} antedates Homer by fifteen hundred years \cite{sandars1972epic}. Yet, its discussion of the nature of humanity captures the attention of global audiences even to this day. The ancient epic discusses pivotal themes that permeate literature even to this day, such as the inevitability of mortality, civil vs. moral law, society’s dichotomous depictions of femininity, isolation vs. companionship, civilization vs. savagery, and ultimately, the meaning of human existence \cite{snigowski2015}.

%
%Gilgamesh is the fifth king of Uruk after the great flood. He built the city's walls. He's two thirds god and one third human. He is the greatest warrior ever to live, but he oppressed his people, etc.. It is the story of Gilgamesh andEnkidu becoming human together.
%
\subsubsection{The \emph{Epic of Gilgamesh}}

According to the standard Akkadian text of the poem, Gilgamesh is the fifth postdiluvian king of Uruk.\footnote{The Sumerian king list cites the reigns of many early rulers to be of superhuman length, so it is difficult to confirm Gilgamesh’s existence as a historical figure. It is thought that the king who inspired the epic lived between 2800 and 2500 BCE.\cite{dalley2000myths}} He is two thirds god and one third human,\footnote{There are conflicting traditions regarding Gilgamesh’s parentage. Gilgamesh inherited his divinity from his mother Ninsum, the goddess Lady Wild Cow. Accounts of Gilgamesh’s father, however, are more varied. The poem itself states Gilgamesh’s father to be Lugulbanda, a shepherd who became king of Uruk two kings before Gilgamesh. Conversely, the king list affirms that Gilgamesh’s father was a \emph{iillu}—a demonic man—who became a priest of Kullab before his kingship. It is possible that Gilgamesh conquered Uruk rather than inheriting the throne, subsequently adopting a famous ruler as his ersatz father. \cite{dalley2000myths}} and though he is humanity’s strongest warrior, he rules tyrannically. In response to the people’s cries, the sky god Anu creates Enkidu, a wild man, to be Gilgamesh’s equal. Encode lives in the forest, in harmony with the animals, but when a hunter reports that Enkidu has foiled his traps and released the animals, the temple prostitute Shamat is sent to seduce Enkidu. After six days and seven nights, Enkidu attempts to return to the forest, but the animals reject him; he has ben inducted into society and thus is no longer of their rank. Thus, Shamat takes Enkidu to a shepherd’s camp to educate him on the requirements of civilization. Meanwhile, Gilgamesh’s mother Ninsum advises Gilgamesh that his dream of a fallen star so massive that even he cannot lift it signifies the eminent arrival of a beloved companion.

At the shepherd’s camp, Enkidu becomes enraged at Gilgamesh’s oppression of his people, and he travels to Uruk in order to intervene at a wedding where Gilgamesh plans to enforce \emph{jus primae noctus}. The two battle fiercely, and Enkidu eventually relents under Gilgamesh’s superior strength. The two feast and become friends, and Gilgamesh proposes that he an Enkidu should attempt to slay Humbaba, the guardian of the Cedar Forest, in pursuit of timeless fame. Despite the warnings of Enkidu and the elders of Uruk, Gilgamesh is not dissuaded, and the two begin preparation for their voyage. Ninsum seeks protection from the sun god Shamash on behalf of the warriors, and she adopts Enkidu as her son before he sets out with Gilgamesh to slay the beast.

The way to the Cedar Forest is long and arduous, so Gilgamesh and Enkidu camp every few nights and perform dream rituals. Gilgamesh has nightmares about natural disasters and fearsome beasts, but Enkidu interprets them as good omens that they will defeat the forest guardian. As they approach the forest, they hear Humbaba roaring, and afraid. They continue on, however, and they battle with Humbaba until Shamash sends thirteen winds to entrap the beast. Hummable begs for his life, offering to make Gilgamesh the king of the forest and to be his slave, but Enkidu urges Gilgamesh not to have mercy, remembering how the dragon insulted them earlier and how he must be lying. Gilgamesh beheads Humbaba with a blow to the neck, and he and Enkidu return home along the Euphrates with the monster's head and timber enough to build a new gate for the temple of Enlil. 

Upon the warriors’ return, the goddess Ishtar proposes marriage to Gilgamesh, but he rejects her because the frequency of other lovers’ demises. Angered, Ishtar demands that her father Anu release Gugalanna, the Bull of Heaven, onto the people of Uruk. Anu refuses, but he relents when Ishtar threatens to raise a plague of zombies to devour the living. Ishtar releases the Bull of Heaven onto Uruk, and it destroys much of the city before Gilgamesh and Enkidu are able to slay it. The sacrifice the beast’s heart to Shamash, and when Ishtar objects, Enkidu throws one of the Bull’s hindquarters at her. 

The people of Uruk celebrate, but Enkidu has a vision of his own demise at the hands of the gods. Even against the will of Shamash, Enkidu is destined to die as punishment for helping slay Humbaba and Gugalanna. Enkidu curses the cedar door to the temple of Enlil, as well as the trapper and Shamat for introducing him to society and thus to mortality. Shamash tells Enkidu that his companionship with Gilgamesh is the direct product of Shamat’s actions and that Gilgamesh will mourn Enkidu well,so Enkidu blesses the prostitute instead. However, Enkidu also dreams that an angel of death captures and forcibly takes him to the Netherworld, a terrifying place where people must eat clay and clothe themselves in bird feathers. Twelve days later, Enkidu dies.

Gilgamesh mourns his companion and orchestrates an elaborate funeral including the dedication a statue in Enkidu’s likeness and a banquet at which sacrifices to the gods of the Netherworld are made.\footnote{There is also mention that a river is to be dammed, suggesting Enkidu’s burial is to take place in a riverbed. This aligns with the description in the Sumerian poem \emph{The Death of Gilgamesh}.} Gilgamesh wanders the wilderness dressed in animal skins, driven mad with grief. In pursuit of eternal life, Gilgamesh seeks out Utnapishtim and his wife, the sole survivors of the Great Flood who were granted immortality by the gods. On his journey, Gilgamesh encounters a pride of lions and prays to the moon god Sin for protection before slaying and skinning them. He travels to the end of the Earth on his journey to the twin peaks of Mt. Mashu. The tunnel through the mountain is guarded by two scorpion men who warn him that no man before him has ever crossed. They allow him to pass, however, and Gilgamesh voyages in darkness along the Road of the Sun for twelve “double days” before reaching the paradisiacal Garden of the Gods. 

On his journey,  Gilgamesh meets Siduri, an alewife who attempts to distract him from his journey by offering to feed, clothe, and pleasure him. He refuses her advances, however, and she directs him to Urshanabi the ferryman. In a spontaneous fit of rage, Gilgamesh demolishes the stone giants that live with Urshanabi, destroying his only means of crossing the poisonous Waters of Death, Urshanabi tells Gilgamesh to fell 120 timbers to use as punting poles, and the two are able to cross the sea to the underworld. When they arrive, Utnapishtim castigates Gilgamesh for seeking eternal life, saying that attempting to overcome the limitations of the human condition can bring only sorrow. 

Gilgamesh asks Utnapishtim how he came to obtain eternal life, and Utnapishtim tells him of how the gods sent a great flood to destroy humanity, but the god Ea warned Utnapishtim and told him to build an arc that could house his entire family, his craftsmen, and all the animals. The tempest raged for six days and nights, and the gods fled to the heavens and lamented the fate of humanity. Utnapishtim docked on a mountain and mourned the people left unsheltered who had turned to clay. He released a dove, a sparrow, and a raven, and when the raven did not return, he disembarked with his family. Utnapishtim makes a sacrifice to the gods, and Ishtar vows never to forget the flood. Enlil is furious that there are survivors, but blesses Utnapishtim and his wife with immortality when Ishtar and Ea chastise him for having caused the devastation and having chosen too draconian a punishment for his people.\footnote{The poem’s account of the deluge aligns with many other flood stories from around the world in which an elite group of people, often a couple, are selected as the only ones worthy of salvation. The most prominent of these in the Western tradition is the Old Testament’s depiction of Noah’s arc.}

Utnapishtim emphasizes the uniqueness of his gift, and he challenges Gilgamesh to stay awake for six days and seven nights, conquering sleep as he would mortality. Gilgamesh becomes drowsy, however, and Utnapishtim’s wife bakes one loaf of bread for every day that Gilgamesh sleeps so that he cannot deny his failure. Defeated, Gilgamesh prepares to return to Uruk. As a parting gift, Utnapishtim tells Gilgamesh of a thorned plant that has the power to bestow the elixir of eternal life. In order to obtain the plant, Gilgamesh must tether stones to his legs and search the bottom of the ocean. Gilgamesh obtains it, cuts the ties keeping the rocks in place, and swims to the surface. He leaves the flower unattended on the bank, however, and a serpent devours it, leaving only it discarded skin in its place. Lamenting the futility of his efforts, Gilgamesh returns to Uruk. Upon seeking the walls of the city, however, he praises their enduring nature.

Fragments of a conclusion exist, but they are disjointed. They include Enkidu’s voyage to the underworld in an attempt to recover some of Gilgamesh’s lost possession and his subsequent entrapment there.

%
%Introduction to modern science fiction and \emph{Doctor Who}
%Science fiction is by nature an experimental genre. It allows its creators to challenge social mores, discussing many ancient themes. 
%

\subsection{The Evolution of Science Fiction}

Science fiction is by nature an experimental genre; it allows creators to merge the realm of the factual with that of the fantastic in order to explore the boundaries of morality. Whether like \emph{Star Trek} it illustrates the utopian future that technological advancements could produce, or like \emph{Frankenstein} demonstrates the potentially cataclysmic effects of the abuse of technology, science fiction ultimately challenges popular conceptions of right and wrong. It is science fiction’s capacity to seamlessly meld contemporary relevance and demonstration of essential human psychology that makes it so alluring to readers globally and intergenerationally. Science fiction addresses fundamental questions of the human condition, and thus inevitably is a vessel for the manifestation of archetypes \cite{reich2013}.

%\emph{Doctor Who} is a long-running BBC television program about a functionally immortal Time Lord who travels through time and space fighting monsters and learning to be human with the help of his companions. The Doctor was forced to ommit genocide in order to end the Time War.
%

\subsubsection{\emph{Doctor Who}}

\emph{Doctor Who} is a long-running BBC television show which first aired on 23 November 1963 \cite{mcdonough2013}. The show tells the story of the Doctor, a centuries old Time Lord who travels through space and time in the TARDIS, a sentient ship that he borrowed never intending to return her. Instead of dying, Time Lords have the ability to renew every cell in their bodies, functionally becoming a different person with each regeneration. As a result, the Doctor has worn more than a dozen faces, each one with unique characteristics and desires.  

Though the Doctor’s often endearing attempts to fit into human society—such as his unique fashion choices and taste in cuisine—make him relatable, his ultimately alien physiology comments on the nature of social isolation; the Doctor is an outsider to human activities, and he will never truly have an equal. However, the Doctor acquires companions as he travels, and he introduces to the wonders of the universe. The Doctor’s companions humanize him on his godlike escapades, maintaining his sanity and keeping him functioning rationally in the face of impossible odds. However, the companions often must meet their demise while traveling, and the Doctor goes through rouge phases following their losses.

The Doctor’s most prominent period of ideological aberrance, however, was during the Time War. The Doctor is to commit genocide in order to end the war between the Time Lord and the Daleks, a mutant warrior race whose express purpose in life is to exterminate all life that is not Dalek. The Doctor is forever scarred by the Time War, as he has to violate his strictly pacifistic ideology in order to protect trillions of innocent casualties throughout the universe. Following the Time War, the Doctor wanders somewhat purposelessly throughout the universe, outwitting corrupt alien intelligences in a futile attempt to regain his moral nobility. Eventually, the Doctor is able to capture his entire home planet of Gallifrey in suspended animation within an oil painting called “Gallifrey Falls No More,” and he finally is assured of the safety of his race.

\section{Comparative Archetypes}
\subsection{The Hero's Journey}
%
%Out of normalcy and into the realm of the supernatural, wherein they must fight monsters/overcome impossible odds before returning with a bounty, home.
%

\subsection{The Wild Man}
%
%Ran with the animals, drank from their streams, introduced to society, rejected by the animals. 
%

\subsection{Femininity and Misogyny}
\subsubsection{The Wise Woman}
%
%TARDIS=Ninsum
%

\subsubsection{The Seductress}
%
%Missy=Prostitute (Not sectualized, but distracted the Doctor from his goal)
%

\subsubsection{The Beloved}
%
%River≠Anyone in the \emph{Epic of Gilgamesh}, because it's mysogynistic
%

\subsection{Immortality and the Afterlfie}
%
%Regenration isn't permenant, just as the snake eats the flower. The afterlife is a physical place, but you cannot thwart it.
%

\section{Analysis}
%
%The Doctor and Gilgamesh are both given ultimate power, but are ultimately not qualified to weild it.
%
%Fatal flaw=hubris
%

\section{Summary and Conclusion}
%
%Relevence restated
%

\section{Acknowledgements}
%
%Greateful acknowledgement is made to Dr. Joe Loughry, Kathryn Schmidt, Megan Gorman, Pamela Reich, Peter Sniegowski, Don Nelson, Lisa Weinberg, Anne Evans, and Kathryn MacNammee for insightful comments and mentorship throughout this process.
%

\bibliographystyle{plain}


\bibliography{references.bib}

\medskip\noindent{\scriptsize Build \input{counters/paper_build_counter.txt}



\end{document}