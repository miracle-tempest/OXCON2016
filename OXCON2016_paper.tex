%
% Paper to be submitted to the 2016 Oxford Comic Con (OXCON 2016),
% Oxford, % England, 5--6 March 2016.
%
% For information on this file please contact Miranda K. Loughry
% at Tel. +1 303 221 4380 (time zone GMT minus 7 hours) or Email:
% miranda.loughry@gmail.com
%

\documentclass[10pt,a4paper]{article}

\usepackage[english,british]{babel}
\usepackage{graphicx}
\usepackage{url}
\usepackage[affil-it]{authblk}

\begin{document}

\title{The Modern \emph{Gilgamesh}: \\
Jungian Archetypes in \emph{Doctor Who}}

\author{Miranda Loughry%
\thanks{Corresponding author's address: \texttt{miranda.loughry@gmail.com}}}
\affil{St Mary's Academy \\
Englewood, Colorado 80113 \\
USA}

\date{\today}

\maketitle

\begin{abstract}
Since the Agricultural Revolution, storytelling has evolved from a means to
caution future generations into a ideal medium to reveal humanity's most
primal fears and desires; stories discussing the essence of humanity, the
inevitability of mortality, humankind's relationship with nature, and the
role of the divine appear throughout history. From the \emph{Epic of
Gilgamesh} to \emph{Doctor Who} and \emph{Forbidden Planet}, many of
humanity's most enduring stories are built on the same fundamental
patterns. Analyzing these cross-cultural, cross-generational motifs through
the lens of modern science fiction reveals powerful commentary on the
nature of humanity. Like Gilgamesh, \emph{Doctor Who} never was about The
Doctor.
\end{abstract}

\section{Introduction}

Some words go here.

\subsection{This is a subsection}

More words\ldots

\newpage

Some more words go here. See Figure \ref{figure:whatever}.

\begin{figure}[!t]
    \centering
	% 'trim' specifies how much to remove from left, bottom, right, and top
	% edges of an A4 size PDF page.
	\includegraphics[width=0.3\textwidth,trim=0 0 117mm 183mm,clip]{graphics/ox_brand_cmyk_pos.pdf}
	\caption{This is the caption.}
	\label{figure:whatever}
\end{figure}

\section{Methodology}

This is a reference \cite{Loughry2013a}.

\section{Results}

Words\ldots

\section{Interpretation}

Words\ldots

\section{Summary and Conclusion}

Words\ldots

\section{Acknowledgements}

More words\ldots

\bibliographystyle{plain}
\bibliography{consolidated_bibtex_file}

\medskip\noindent{\scriptsize Build \input{counters/paper_build_counter.txt}

\end{document}

