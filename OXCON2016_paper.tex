%
% Paper to be submitted to the 2016 Oxford Comic Con (OXCON 2016),
% Oxford, % England, 5--6 March 2016.
%
% For information on this file please contact Miranda K. Loughry
% at Tel. +1 303 221 4380 (time zone GMT minus 7 hours) or Email:
% miranda.loughry@gmail.com
%

\documentclass[10pt,a4paper]{article}

\usepackage[english,british]{babel}
\usepackage{graphicx}
\usepackage{url}
\usepackage[affil-it]{authblk}

\begin{document}

\title{The Modern \emph{Gilgamesh}: \\
Jungian Archetypes in \emph{Doctor Who}}

\author{Miranda Loughry%
\thanks{Corresponding author's address: \texttt{miranda.loughry@gmail.com}}}
\affil{St Mary's Academy \\
Englewood, Colorado 80113 \\
USA}

\date{\today}

\maketitle

\begin{abstract}
Since the Agricultural Revolution, storytelling has evolved from a means to
caution future generations into a ideal medium to reveal humanity's most
primal fears and desires; stories discussing the essence of humanity, the
inevitability of mortality, humankind's relationship with nature, and the
role of the divine appear throughout history. From the \emph{Epic of
Gilgamesh} to \emph{Doctor Who} and \emph{Forbidden Planet}, many of
humanity's most enduring stories are built on the same fundamental
patterns. Analyzing these cross-cultural, cross-generational motifs through
the lens of modern science fiction reveals powerful commentary on the
nature of humanity. Like Gilgamesh, \emph{Doctor Who} never was about The
Doctor.
\end{abstract}

\section{Introduction}
    Archetypes are cross-cultural, cross-generational motifs that exist as manifestations of the collective unconscious. They are most prominent in dreams, wherein the subconscious is the chief force at work. Carl Jung, a protege of the pioneering psychoanalyst Sigmund Freud, wrote extensively about them.
   
   Civilization began in the Fertile Crescent. One the greatest early civilizations was that of Mesopotamia. In the \emph{Epic of Gilgamesh}, one of their greatest/earliest long poems, the Mesopotamians wrote about themes that are still relevant today (the inevitability of mortality; the role of the divine, as well as moral/divine vs. civil law; the duality of depictions of femininity; civilization vs. savagrey, etc.). 
   
   Science fiction is by nature an experimental genre. It is a place for creators to test new ideas, challanging social mores and norms (\emph{Frankenstein}, \emph{Fahrenheit 451}, \emph{Nineteen Eighty-Four}). Thus, it is an ideal medium for archetypes to manifest. \emph{Doctor Who} exemplifies this.


\subsection{Thesis}
The Doctor and Gilgamesh are manifestations of the same archetype.


%Body

\section{The Hero's Journey}
    Must venture from the realm of normalcy through an extraordinary land, fighting monsters, etc.. Greek: anagnorisis and peripety.
    
    
 \section{Comparative Archetypes}     
 
\subsection{The Wildman}
    Gilgamesh's companion Enkidu and the Doctor's companions both fulfil the archetype of the wildman, in that they are both naive (Enkidu of oldness and the companions of the greater community of the universe). They represent humanity's primal past (commentary on humanity's still savage nature, especially with the companions), and their induction into society (Enkidu by Gilgamesh through the prostitute and the companions through the Doctor's travels) directly results in their deaths (Enkidu damned by the gods for killing Humbaba, the companions in various ways: Amy and Rory by Weeping Angels, zapped into the past and made to live to death; Danny by Cybermen; Clara by the raven; many one-episode companions like Rita, Astrid, etc.) or else their demises (Rose trapped eternally in a parallel universe; Donna forced to forget her time with the Doctor; Me forced to an eternal).
    
\subsection{Depictions of Femininity}
\subsubsection{The Seductress}
    The prostitute/the barmaid, and Missy 
\subsubsection{The Wise Woman}
    Ninsum and the TARDIS
\subsection{The Afterlife}
    The flower of eternal life and regeneration. Both aren't absolute. Both must be curated/moderated, either by the snake or by the Sisterhood.

This is a reference \cite{Loughry2013a}.

Words\ldots

\section{Interpretation}

Words\ldots

\section{Summary and Conclusion}

Words\ldots

\section{Acknowledgements}

More words\ldots

\bibliographystyle{plain}
\bibliography{consolidated_bibtex_file}

\medskip\noindent{\scriptsize Build \input{counters/paper_build_counter.txt}

\end{document}

